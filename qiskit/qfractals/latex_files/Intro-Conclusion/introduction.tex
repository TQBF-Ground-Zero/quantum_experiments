\chapter*{Introduction}
\label{chapter:introduction}
\addcontentsline{toc}{chapter}{Introduction}
\markboth{Introduction}{Introduction}

De tout temps, les doctorants et doctorantes ont voulu rédiger leur thèse en \LaTeX $\,$ sans avoir à perdre trop de temps sur l'élaboration d'un template fonctionnel. Malheureusement, faire un template clef-en-main qui réponde aux attentes de tout le monde est illusoire. Néanmoins, je vous partage le mien, parce qu'on ne sait jamais, peut-être qu'il vous conviendra et que vous n'aurez pas à modifier trop de choses dedans pour avoir un rendu qui vous plaise ! (Et au passage je discute quelques points souvent utiles pour les thèses de sciences : comment gérer les unités, et comment faire des figures avec des sous-figures imbriquées qui sont citables dans le texte.)